\section{Methodik}

\subsection{Auswahl geeigneter Datenbanken und Suchbegriffe}

Um relevante Literaturquellen für die vorliegende Arbeit zu identifizieren, wurde eine systematische Literaturrecherche durchgeführt. Die Auswahl geeigneter Datenbanken und die Festlegung der Suchbegriffe waren entscheidend für die Effektivität der Recherche. In diesem Abschnitt werden die gewählten Datenbanken und Suchbegriffe detailliert erläutert.

\paragraph{Datenbanken}

Die Auswahl der Datenbanken erfolgte basierend auf ihrer Relevanz für das Forschungsthema. Dabei wurden folgende Datenbanken berücksichtigt:

\begin{itemize}
    \item \textbf{Google Scholar:} Google Scholar ist eine frei zugängliche Suchmaschine, die eine breite Palette wissenschaftlicher Literatur abdeckt, einschließlich Artikel, Konferenzbeiträge, Bücher und Patente. Sie bietet eine umfassende Abdeckung verschiedener Disziplinen und ermöglicht eine einfache Suche nach wissenschaftlichen Arbeiten.

    \item \textbf{Semantic Scholar:} Semantic Scholar verwendet fortschrittliche Technologien, um wissenschaftliche Literatur zu durchsuchen und relevante Informationen zu extrahieren. Die Plattform legt besonderen Wert auf semantische Analysen und bietet Funktionen wie die Identifikation von Schlüsselbegriffen und Beziehungen zwischen verschiedenen Werken.

    \item \textbf{IEEE Xplore:} IEEE Xplore ist eine digitale Bibliothek, die auf Ingenieur- und Informatikliteratur spezialisiert ist. Sie enthält eine Fülle von wissenschaftlichen Artikeln, Konferenzberichten, Standards und technischen Zeitschriften, die von der IEEE (Institute of Electrical and Electronics Engineers) veröffentlicht wurden.
\end{itemize}

Die genannten Datenbanken wurden aufgrund ihrer umfassenden Abdeckung im Bereich Computational Fluid Dynamics ausgewählt und liefern somit umfangreiche Suchergebnisse für die systematische Literaturrecherche.

\paragraph{Suchbegriffe}

Um relevante Literaturquellen für die vorliegende Arbeit zu identifizieren, wurden spezifische Suchbegriffe in den ausgewählten Datenbanken verwendet. Die Suche erfolgte in Deutsch und Englisch, um sicherzustellen, dass eine breite Palette relevanter Arbeiten erfasst wird. Hier sind die Haupt-Suchbegriffe:

\begin{table}[ht]
    \centering
    \small
    \begin{tabular}{p{0.45\textwidth}|p{0.45\textwidth}}
        \toprule
        \textbf{Deutsch} & \textbf{Englisch} \\
        \midrule
        Herausforderungen beim Diskretisieren in CFD & Challenges in CFD discretization/discretisation \\
        \midrule
        Herausforderungen beim Lösen der Druckgleichung in CFD & Challenges in solving the pressure equation in CFD \\
        \midrule
        Lösungsansätze um die Druckgleichung in CFD effizient zu lösen & Approaches to efficiently solve the pressure equation in CFD \\
        \midrule
        Anwendung von Deep Learning in der Strömungsdynamik & Application of Deep Learning in fluid dynamics \\
        \midrule
        Druckgleichung in CFD mithilfe von Deep Learning lösen & Solving the pressure equation in CFD using Deep Learning \\
        \bottomrule
    \end{tabular}
    \caption{Suchbegriffe in Deutsch und Englisch}
    \label{tab:search_terms}
\end{table}

Die Kombination dieser Suchbegriffe ermöglichte eine gezielte Suche nach Literaturquellen, die eng mit dem Forschungsthema verbunden sind.

\subsection{Einschluss- und Ausschlusskriterien}
\textbf{Einschlusskriterien:}

\begin{enumerate}
  \item \textbf{Relevanz für die Navier-Stokes Gleichung} Literaturquellen sollen sich direkt mit der numerischen Lösung der Navier-Stokes Gleichung und insbesondere mit der Druckgleichung befassen. Arbeiten, die eine klare Verbindung zwischen Deep Learning und der Strömungsdynamik herstellen, sind von Interesse.
  
  \item \textbf{Aktualität} Die ausgewählten Quellen sollten aktuell sein, vorzugsweise \textbf{innerhalb der letzten fünf Jahre veröffentlicht}. Dies gewährleistet, dass die Literatur die neuesten Entwicklungen und Fortschritte in der Anwendung von Deep Learning in der Strömungsdynamik abdeckt.
  
  \item \textbf{Wissenschaftliche Qualität} Die ausgewählten Arbeiten müssen einem qualitativen wissenschaftlichen Standard entsprechen. Hierzu gehören peer-reviewte Zeitschriftenartikel, Konferenzbeiträge oder akademische Bücher, die von Experten auf dem Gebiet begutachtet wurden.
  
  \item \textbf{Praxisrelevanz} Die Literaturquellen sollten einen klaren Bezug zur Praxis haben und aufzeigen, wie die Anwendung von Deep Learning die numerische Lösung der Navier-Stokes Gleichung in realen Anwendungsfällen verbessern kann.
\end{enumerate}

\textbf{Ausschlusskriterien:}

\begin{enumerate}
  \item \textbf{Irrelevanz für die Navier-Stokes Gleichung} Literatur, die sich nicht direkt mit der Navier-Stokes Gleichung oder deren numerischer Lösung befasst, wird ausgeschlossen. Arbeiten, die sich nur oberflächlich mit dem Thema beschäftigen oder es nicht in einem relevanten Kontext diskutieren, werden nicht berücksichtigt.
  
  \item \textbf{Mangelnde Aktualität} Veraltete Literatur, die nicht die neuesten Fortschritte in der Anwendung von Deep Learning in der Strömungsdynamik reflektiert, wird ausgeschlossen.
  
  \item \textbf{Geringe wissenschaftliche Qualität} Quellen ohne Peer-Review oder aus zweifelhaften wissenschaftlichen Zeitschriften werden ausgeschlossen, um die Qualität und Glaubwürdigkeit der Informationen sicherzustellen.
  
  \item \textbf{Fehlende Praxisrelevanz} Arbeiten, die keine klaren Beispiele oder Anwendungen für die Verbesserung der numerischen Lösung der Navier-Stokes Gleichung durch Deep Learning bieten, werden ausgeschlossen.
\end{enumerate}

Die Anwendung dieser Kriterien soll sicherstellen, dass die ausgewählte Literatur eine solide Basis für die Literaturrecherche bildet und relevante, qualitativ hochwertige Informationen liefert.
