\section{Grundlagen}
\subsection{Inkompressible Navier-Stokes Gleichung}
\subsubsection{Herleitung und mathematische Formulierung}
Die inkompressible Navier-Stokes Gleichung ist eine partielle Differentialgleichung, die die Erhaltung von Masse und Impuls in einem strömenden Fluid beschreibt. Die Herleitung beginnt mit der Anwendung des Gesetzes der Massenerhaltung auf ein infinitesimales Fluidvolumen. Das resultierende Kontinuitätsgesetz lautet:

\begin{equation}
\frac{\partial \rho}{\partial t} + \nabla \cdot (\rho \mathbf{u}) = 0
\end{equation}

Dabei ist $\rho$ die Dichte des Fluids und $\mathbf{u}$ ist die Geschwindigkeitsvektor. Die zweite Gleichung der Bewegung ergibt sich aus der Anwendung des Impuls-Erhaltungsgesetzes und lautet:

\begin{equation}
\rho \left(\frac{\partial \mathbf{u}}{\partial t} + (\mathbf{u} \cdot \nabla)\mathbf{u}\right) = -\nabla p + \mu \nabla^2 \mathbf{u} + \mathbf{f}
\end{equation}

Hierbei steht $p$ für den Druck, $\mu$ für die dynamische Viskosität des Fluids und $\mathbf{f}$ für externe Kräfte pro Einheitsvolumen.

Für den Fall der Inkompressibilität, bei dem die Dichte konstant ist ($\frac{\partial \rho}{\partial t} = 0$), vereinfacht sich die Kontinuitätsgleichung zu:

\begin{equation}
\nabla \cdot \mathbf{u} = 0
\end{equation}

Die Kombination der inkompressiblen Kontinuitätsgleichung mit der zweiten Gleichung der Bewegung ergibt die inkompressible Navier-Stokes Gleichung:

\begin{equation}
\frac{\partial \mathbf{u}}{\partial t} + (\mathbf{u} \cdot \nabla)\mathbf{u} = -\frac{1}{\rho}\nabla p + \nu \nabla^2 \mathbf{u} + \mathbf{f}
\end{equation}

Dabei ist $\nu$ die kinematische Viskosität, definiert als $\nu = \frac{\eta}{\rho}$.

Die inkompressible Navier-Stokes Gleichung kann durch Hinzufügen der Druckgleichung in ihrer vollständigen Form präsentiert werden. Die Druckgleichung wird durch Anwendung des Divergenzoperators auf die inkompressible Kontinuitätsgleichung abgeleitet und lautet:

\begin{equation}
\nabla^2 p = -\frac{\rho}{\nu} \cdot \left(\nabla \cdot \mathbf{u} \right)
\end{equation}

Hierbei steht $\nabla^2 p$ für den Laplace-Operator des Drucks $p$, $\rho$ ist die Dichte des Fluids, $\nu$ ist die kinematische Viskosität, und $\nabla \cdot \mathbf{u}$ repräsentiert den Divergenzoperator angewandt auf das Geschwindigkeitsvektor $\mathbf{u}$.

Die Druckgleichung ist ein entscheidender Bestandteil der inkompressiblen Navier-Stokes Gleichung und stellt häufig den anspruchsvollsten Teil dar, wenn es darum geht, numerische Lösungen für Strömungsdynamikprobleme zu finden. Die Lösung dieser Gleichung erfordert spezielle numerische Techniken, da der Druck nicht direkt gemessen wird und dessen Gradient für die Berechnung von Geschwindigkeitsfeldern notwendig ist.

Diese Gleichungen bilden die Grundlage für die numerische Lösung von Strömungsdynamikproblemen unter Verwendung von Finite-Differenzen-Methoden. 

\subsubsection{Physikalische Bedeutung der Gleichung}
Lorem Ipsum

\subsection{Numerische Strömungssimulation mit Finite Differenzen}
\subsubsection{Grundlagen der Finite-Differenzen-Methode} 
Lorem Ipsum

\subsubsection{Diskretisierungstechnicken und Lösungsverfahren speziell für Finite Differenzen} 
Lorem Ipsum

\subsection{Deep Learning}
\subsubsection{Grundprinzipien von Deep Learning} 
Lorem Ipsum doloris

\subsubsection{Anwendungen von Deep Learning in wissenschaftlichen Bereichen} Lorem Ipsum doloris