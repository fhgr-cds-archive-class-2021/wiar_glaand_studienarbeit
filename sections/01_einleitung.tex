\section{Einleitung}
\subsection{Hintergrund}
\paragraph{Bedeutung der Navier-Stokes Gleichung in der Strömungsdynamik}
Die Navier-Stokes Gleichung spielt eine zentrale Rolle in der Strömungsdynamik und bietet einen fundamentalen Rahmen zur Beschreibung von Flüssigkeits- und Gasbewegungen. Diese partielle Differentialgleichung, die aus den Gesetzen der Masse- und Impulserhaltung hergeleitet wird, ermöglicht die mathematische Modellierung und Simulation von Strömungen unter verschiedenen Bedingungen. \parencite{simon_schneiderbauer_da8e2d1d}

Die Bedeutung dieser Gleichung liegt in ihrer Fähigkeit, komplexe Strömungsphänomene zu charakterisieren, von laminaren Strömungen in Rohrleitungen bis hin zu turbulenten Wirbeln in der Atmosphäre oder Ozeanen. Durch die Berücksichtigung von Druck, Viskosität und externen Kräften ermöglicht die Navier-Stokes Gleichung die präzise Vorhersage von Strömungsverhalten und -mustern. \parencite{p_g_drazin__n_riley_482fe063}

In der Praxis wird die inkompressible Form der Gleichung oft bevorzugt, da sie für viele strömungsrelevante Anwendungen geeigneter ist. Sie gilt insbesondere für Flüssigkeiten oder Gase mit vernachlässigbarer Dichteänderung während der Strömung, was bei vielen realen Szenarien der Fall ist. \parencite{p_g_drazin__n_riley_482fe063}

\paragraph{Herausforderungen bei der Lösung der Navier-Stokes Gleichung}
Die numerische Lösung der Navier-Stokes Gleichung ist keine triviale Aufgabe. Der Algorithmus besteht aus verschiedenen Schritten, wie der Diskretisierung, der Festlegung von Randbedingungen, der Berechnung des Geschwindigkeitsvektors und der Lösung der Druckgleichung. Die grösste Herausforderung der Gleichung liegt dabei im Lösen der Druckgleichung. Nach jedem Zeitschritt muss diese individuell gelöst werden, wie in der mathematischen Gleichung \ref{eq:poisson_pressure} dargestellt. \parencite{stefan_turek_3ee2e7b8}

\begin{equation}
\nabla^2 p = -\frac{\rho}{\nu} \cdot \left(\nabla \cdot \mathbf{u} \right)\label{eq:poisson_pressure}
\end{equation}

Es existieren verschiedene numerische Verfahren, um mathematische Gleichungen zu lösen, wie direkte und iterative Gleichungslöser. Eine detaillierte Diskussion über die Vor- und Nachteile jedes Gleichungslösers geht über den Umfang dieser Arbeit hinaus. Im Wesentlichen ist es jedoch wichtig, dass ein Gleichungslöser als effizient gilt, wenn er möglichst wenige Rechenschritte benötigt, um eine bestimmte Residuumstoleranz zu erreichen.

Das Hauptproblem besteht darin, dass es je nach Fluidproblem schwierig ist, den richtigen Gleichungslöser zu wählen und ihn so zu konfigurieren, dass die Residuumswerte der Druckgleichung möglichst stabil bleiben. Bei iterativen Gleichungslösern befindet sich die Lösung auf beiden Seiten der Gleichung, und es kann vorkommen, dass das Residuum plötzlich ansteigt. Solche Situationen möchte man vermeiden. Es gibt verschiedene Ansätze, um dies zu verhindern, wie beispielsweise die Verwendung eines idealen Startwerts. Die Schwierigkeit dabei besteht darin, den richtigen idealen Startwert zu bestimmen. Einige Ansätze schlagen vor, den Nullvektor als Startwert zu verwenden oder mit einfacheren Gleichungslösern einen Schritt zu berechnen, um den Startwert zu erhalten. In letzter Zeit versuchen Forscher auch, mithilfe von Deep Learning den idealen Startwert von der Druckgleichung zu inferieren.

\paragraph{Deep Learning als nichtlinearer Operator}
In den letzten Jahren hat Deep Learning in verschiedenen wissenschaftlichen Disziplinen bahnbrechende Fortschritte ermöglicht. Es zeichnet sich nicht nur durch seine Fähigkeit aus, komplexe Zusammenhänge vorherzusagen, sondern auch dadurch, dass es als nichtlinearer Operator fungieren kann. Die Interaktion zwischen den Neuronen in neuronalen Netzen ermöglicht es, Verbindungen zwischen den Daten herzustellen und daraus Anwendungen abzuleiten. Dieser Mechanismus kann auch auf die Druckgleichung angewendet werden, wie von \textbf{Forschern et al.} in ihrer Arbeit gezeigt wurde. Basierend auf diesen Hinweisen dient die Motivation für diese Untersuchung dazu, die Unterstützung von Deep Learning bei der Lösung der Druckgleichung in der Navier-Stokes-Gleichung genauer zu untersuchen.

\subsection{Forschungsfrage}
Aus den in der Einleitung a
ufgezeigten Problemen und Möglichkeiten ergibt sich die Frage, \textbf{inwieweit Deep Learning als Unterstützung bei der Lösung der Druckgleichung eingesetzt werden kann.} Dabei ist zu betonen, dass Deep Learning nicht den vollständigen Gleichungslöser ersetzen soll, sondern vielmehr Teile davon beschleunigen soll. Am Ende muss sichergestellt werden, dass die Lösung korrekt ist, da die Genauigkeit solcher Lösungen lebenskritisch sein kann.

\subsection{Zielsetzung}
Damit die Fragestellung beantwortet werden kann, soll diese Arbeit anhand einer systematischer Literaturrecherce folgende Ziele verfolgen:

\begin{enumerate}
    \item \textbf{Analyse der aktuellen Herausforderungen bei der numerischen Lösung der Navier-Stokes Gleichung} - Eine eingehende Untersuchung der numerischen Herausforderungen im Zusammenhang mit der Navier-Stokes Gleichung, insbesondere im Kontext der Druckgleichung. Dies umfasst die Identifikation von Schwierigkeiten, möglichen Lösungsansätzen und aktuellen Fortschritten.
    \item \textbf{Bewertung von Deep Learning als potenzielle Unterstützung für die Lösung der Druckgleichung} - Durch die Literaturrecherche soll die Rolle von Deep Learning als nichtlinearer Operator in der Strömungsdynamik bewertet werden, insbesondere im Hinblick auf seine Anwendung zur Beschleunigung der Druckgleichungslösung. Dies beinhaltet die Erforschung bestehender Ansätze und deren Wirksamkeit.
    \item \textbf{Zusammenfassung und Kategorisierung relevanter Literaturquellen} - Eine systematische Zusammenstellung und Kategorisierung von relevanten wissenschaftlichen Arbeiten, Studien und Forschungsberichten, die sich mit der Anwendung von Deep Learning zur Unterstützung bei der Lösung der Druckgleichung in der Navier-Stokes-Gleichung beschäftigen. Hierbei liegt der Fokus auf der Erfassung unterschiedlicher Forschungsansätze und ihrer Ergebnisse.
\end{enumerate}