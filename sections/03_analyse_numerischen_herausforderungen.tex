\section{Analyse der numerischen Herausforderungen}

\subsection{Generelle Herausforderungen}
Die Herausforderungen bei der numerischen Lösung der Navier-Stokes Gleichung, insbesondere im Kontext des Diskretisierungsfehlers, werden von verschiedenen Autoren beleuchtet. Tyson betont die Schwierigkeiten aufgrund der Unbekanntheit der exakten kontinuierlichen Lösung, der Notwendigkeit asymptotischer Gitterkonvergenz und der Herausforderungen bei der Reduktionsfehlerschätzung \parencite{tysonHigherorderErrorEstimation2019}. Mavriplis hebt die Vorteile höherer Diskretisierungsordnungen hervor, weist jedoch auch auf höhere Kosten und Kompromisse hin \parencite{mavriplisProgressCFDDiscretizations2019}. Schaefer unterstreicht komplexe Aspekte der Diskretisierungsfehleinschätzung, darunter glückliche Kompensationen und nicht-monotone Konvergenz \parencite{schaeferApplicationCFDUncertainty2019}. Tominaga betont die Komplexität der Gitterempfindlichkeit, Unsicherheitsschätzung, numerischen Viskosität, Auflösungsqualität und Qualitätsbewertung in CFD-Simulationen \parencite{tominagaAccuracyCFDSimulations2023}. Diese Erkenntnisse unterstreichen die anspruchsvolle Natur der Diskretisierungsfehlerbewältigung in der Strömungsdynamik.

\subsection{Betonung auf Schwierigkeiten bei der Druckgleichung}
Lorem Ipsum

\subsection{Identifikation möglicher Lösungsansätze und aktueller Fortschritte}
Lorem Ipsum