\section{Analyse der numerischen Herausforderungen}

\subsection{Generelle Herausforderungen}
Die Herausforderungen bei der numerischen Lösung der Navier-Stokes Gleichung, insbesondere im Kontext des Diskretisierungsfehlers, werden von verschiedenen Autoren beleuchtet. Tyson betont die Schwierigkeiten aufgrund der Unbekanntheit der exakten kontinuierlichen Lösung, der Notwendigkeit asymptotischer Gitterkonvergenz und der Herausforderungen bei der Reduktionsfehlerschätzung \parencite{tysonHigherorderErrorEstimation2019}. Mavriplis hebt die Vorteile höherer Diskretisierungsordnungen hervor, weist jedoch auch auf höhere Kosten und Kompromisse hin \parencite{mavriplisProgressCFDDiscretizations2019}. Schaefer unterstreicht komplexe Aspekte der Diskretisierungsfehleinschätzung, darunter glückliche Kompensationen und nicht-monotone Konvergenz \parencite{schaeferApplicationCFDUncertainty2019}. Tominaga betont die Komplexität der Gitterempfindlichkeit, Unsicherheitsschätzung, numerischen Viskosität, Auflösungsqualität und Qualitätsbewertung in CFD-Simulationen \parencite{tominagaAccuracyCFDSimulations2023}. Diese Erkenntnisse unterstreichen die anspruchsvolle Natur der Diskretisierungsfehlerbewältigung in der Strömungsdynamik.

\subsection{Betonung auf Schwierigkeiten bei der Druckgleichung}
Die Autoren des Papiers \parencite{frischMeasuringComparingScaling2015} haben keine expliziten Herausforderungen im Zusammenhang mit der Lösung der Druckgleichung aufgeführt. Dennoch betonten sie die kritische Bedeutung der effizienten Lösung der Druck-Poisson-Gleichung, da über 90 \% der Simulationszeit dieser Aufgabe gewidmet wird. Die vorgestellte mehrgitterartige Druck-Poisson-Lösung basiert auf geometrischen Mehrgitter-Lösern und zeigt sich als effektive Methode zur Bewältigung elliptischer partieller Differentialgleichungen, wie sie in der Poisson-Gleichung auftreten.

In Bezug auf die Herausforderungen bei der Druckgleichungsbewältigung wurden in einer anderen Studie \parencite{heinzMathematicalSolutionComputational2023} signifikante Schwierigkeiten hervorgehoben. Dazu zählen die Notwendigkeit der Berücksichtigung turbulenter Diffusionsterme, Unsicherheiten bei der Simulation auf groben Gittern und erhebliche Vorhersageunsicherheiten aufgrund vieler einstellbarer Simulationseinstellungen. Diese Herausforderungen resultieren aus der konzeptionellen Natur des Problems, der Unklarheit über wesentliche Fragen und der begrenzten Validierungsmöglichkeiten aufgrund fehlender experimenteller Daten für hohe Reynolds-Zahlen.

Ein weiterer Blick auf die Herausforderungen im Zusammenhang mit der Lösung der Druckgleichung in der numerischen Strömungsmechanik wird in der Arbeit von \parencite{jasakPracticalComputationalFluid2020} geworfen. Hier werden die nichtdiagonal dominante Druck-Geschwindigkeitsmatrix, die Linearisierung der nichtlinearen Impulsgleichung, das Sattelpunktproblem und die Notwendigkeit stärkerer Löser und Glätter als herausfordernde Aspekte beleuchtet. Diese Komplexität und Nichtlinearität erfordern fortgeschrittene numerische Methoden und iterative Lösungsansätze für eine effektive Bewältigung.

\subsection{Identifikation möglicher Lösungsansätze}

Die vorgeschlagene mathematische Lösung des Computational Fluid Dynamics (CFD)-Dilemmas setzt auf die Entwicklung innovativer Simulationsmethoden basierend auf exakter Mathematik. Diese Methoden versprechen eine effizientere und kontrollierte Auflösungsfähigkeit sowie signifikante Kostenreduktionen im Vergleich zu bestehenden Ansätzen. Die Autoren favorisieren dabei die Verwendung von minimalen Fehler-Simulationsmethoden, bei denen die Druckgleichung durch einen Variationsansatz gelöst wird, der den Fehler zwischen der numerischen und der exakten Lösung minimiert. Diese Herangehensweise ermöglicht eine effiziente und präzise Auflösung turbulenter Strömungen auf vergleichsweise groben Gittern, ohne aufwendige Anpassungen der Simulationseinstellungen oder Validierungen der Ergebnisse \parencite{heinzMathematicalSolutionComputational2023}.

Zur effizienten Lösung der Druckgleichung in der numerischen Strömungsmechanik schlagen die Autoren verschiedene Ansätze vor, darunter Schur-Vorkonditionierung, blockgekoppelte implizite Techniken, iterative Lösungsmethoden und den Einsatz stärkerer Löser und Glätter \parencite{jasakPracticalComputationalFluid2020}.